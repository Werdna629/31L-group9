%%%%%%%%%%%%%%%%%%%%%%%%%%%%%%%%%%%%%%%%%
% Short Sectioned Assignment
% LaTeX Template
% Version 1.0 (5/5/12)
%
% This template has been downloaded from:
% http://www.LaTeXTemplates.com
%
% Original author:
% Frits Wenneker (http://www.howtotex.com)
%
% License:
% CC BY-NC-SA 3.0 (http://creativecommons.org/licenses/by-nc-sa/3.0/)
%
%%%%%%%%%%%%%%%%%%%%%%%%%%%%%%%%%%%%%%%%%

%----------------------------------------------------------------------------------------
%	PACKAGES AND OTHER DOCUMENT CONFIGURATIONS
%----------------------------------------------------------------------------------------

\documentclass[paper=letter, fontsize=11pt]{scrartcl} % A4 paper and 11pt font size

\usepackage[T1]{fontenc} % Use 8-bit encoding that has 256 glyphs
\usepackage{fourier} % Use the Adobe Utopia font for the document - comment this line to return to the LaTeX default
\usepackage[english]{babel} % English language/hyphenation
\usepackage{amsmath,amsfonts,amsthm} % Math packages
\usepackage{graphicx}
\usepackage{lipsum} % Used for inserting dummy 'Lorem ipsum' text into the template
\usepackage{hyperref}

\usepackage{sectsty} % Allows customizing section commands
\allsectionsfont{\centering \normalfont\scshape} % Make all sections centered, the default font and small caps

\usepackage{fancyhdr} % Custom headers and footers
\pagestyle{fancyplain} % Makes all pages in the document conform to the custom headers and footers
\fancyhead{} % No page header - if you want one, create it in the same way as the footers below
\fancyfoot[L]{} % Empty left footer
\fancyfoot[C]{} % Empty center footer
\fancyfoot[R]{\thepage} % Page numbering for right footer
\renewcommand{\headrulewidth}{0pt} % Remove header underlines
\renewcommand{\footrulewidth}{0pt} % Remove footer underlines
\setlength{\headheight}{13.6pt} % Customize the height of the header

\numberwithin{equation}{section} % Number equations within sections (i.e. 1.1, 1.2, 2.1, 2.2 instead of 1, 2, 3, 4)
\numberwithin{figure}{section} % Number figures within sections (i.e. 1.1, 1.2, 2.1, 2.2 instead of 1, 2, 3, 4)
\numberwithin{table}{section} % Number tables within sections (i.e. 1.1, 1.2, 2.1, 2.2 instead of 1, 2, 3, 4)

\setlength\parindent{0pt} % Removes all indentation from paragraphs - comment this line for an assignment with lots of text

%----------------------------------------------------------------------------------------
%	TITLE SECTION
%----------------------------------------------------------------------------------------

\newcommand{\horrule}[1]{\rule{\linewidth}{#1}} % Create horizontal rule command with 1 argument of height

\title{	
\normalfont \normalsize 
\textsc{University of California Irvine} \\  % Your university, school and/or department name(s)
\textsc{Course: Introduction to Digital Logic Lab (31L) Fall 2015} \\ [25pt]
\horrule{0.5pt} \\[0.4cm] % Thin top horizontal rule
\huge Lab 0 Report\\ % The assignment title
\horrule{2pt} \\[0.5cm] % Thick bottom horizontal rule
}

\author{Kelvin Phan \\ Student ID: 51197373
	\and
	Patrick Skoury \\ Student ID: 75202200 % Your name
	\and
	Aaron Liao \\ Student ID: 90811748 % Your name
}

\date{\large\today} % Today's date or a custom date

\begin{document}

\maketitle % Print the title

%----------------------------------------------------------------------------------------
%	PROBLEM 1
%----------------------------------------------------------------------------------------

\section{Lessons Learned}

%------------------------------------------------

In this lab, we learned how to compile VHDL code, elaborate and optimize a System Verilog test bench, and simulate the waveforms produced by the test bench. 

\section{Logs}

\subsection{Compilation Log}

\begin{verbatim}
ncvhdl: 06.20-s015: (c) Copyright 1995-2009 Cadence Design Systems, Inc.
TOOL:	ncvhdl	06.20-s015: Started on Oct 01, 2015 at 19:34:18 PDT
ncvhdl
    orgate_tb.vhd
    -messages

orgate_tb.vhd:
	errors: 0, warnings: 0
WORK.ORGATE_TB (entity):
	streams: 1, words: 9
WORK.ORGATE_TB:ORGATE_TB (architecture):
	streams: 1, words: 31
TOOL:	ncvhdl	06.20-s015: Exiting on Oct 01, 2015 at 19:34:19 PDT  (total: 00:00:01)

\end{verbatim}

\subsection{Elaboration Log}

\begin{verbatim}
ncelab: 06.20-s015: (c) Copyright 1995-2009 Cadence Design Systems, Inc.
TOOL:	ncelab	06.20-s015: Started on Oct 01, 2015 at 19:34:57 PDT
ncelab
    orgate_tb:orgate_tb
    -messages
    -access +r

	Elaborating the design hierarchy:
ncelab: *W,CUDEFB: default binding occurred for component instance (:orgate_tb(orgate_tb):or1) with design unit (WORK.ORGATE:BEHAVIORAL).
	Building instance specific data structures.
	Design hierarchy summary:
		           Instances  Unique
		Components:        3       2
		Default bindings:  1       -
		Processes:         2       2
		Signals:           3       3
	Writing initial simulation snapshot: WORK.ORGATE_TB:ORGATE_TB
TOOL:	ncelab	06.20-s015: Exiting on Oct 01, 2015 at 19:34:58 PDT  (total: 00:00:01)

\end{verbatim}

\subsection{Simulation Log}

\begin{verbatim}
ncsim: 06.20-s015: (c) Copyright 1995-2009 Cadence Design Systems, Inc.
ncsim: *W,DLNOHV: Unable to find an 'hdl.var' file to load in.
TOOL:	ncsim	06.20-s015: Started on Oct 01, 2015 at 16:24:41 PDT
ncsim
    orgate_tb
    -input run.cmd
    -messages

Loading snapshot work.orgate_tb:orgate_tb .................... Done
ncsim> database -open waves -into waves.shm -default
Created default SHM database waves
ncsim> probe -create -shm -all -variables -depth all
Created probe 1
ncsim> run 50ns
Ran until 50 NS + 0
ncsim> exit
TOOL:	ncsim	06.20-s015: Exiting on Oct 01, 2015 at 16:24:58 PDT  (total: 00:00:17)

\end{verbatim}

\subsection{Simulation Log (Questasim)}

\begin{verbatim}

# vsim -l simulation.log -do "sim.do" -c orgate_tb_opt 
# Start time: 16:02:10 on Oct 03,2015
# //  Questa Sim-64
# //  Version 10.4c linux_x86_64 Jul 19 2015
# //
# //  Copyright 1991-2015 Mentor Graphics Corporation
# //  All Rights Reserved.
# //
# //  THIS WORK CONTAINS TRADE SECRET AND PROPRIETARY INFORMATION
# //  WHICH IS THE PROPERTY OF MENTOR GRAPHICS CORPORATION OR ITS
# //  LICENSORS AND IS SUBJECT TO LICENSE TERMS.
# //  THIS DOCUMENT CONTAINS TRADE SECRETS AND COMMERCIAL OR FINANCIAL
# //  INFORMATION THAT ARE PRIVILEGED, CONFIDENTIAL, AND EXEMPT FROM
# //  DISCLOSURE UNDER THE FREEDOM OF INFORMATION ACT, 5 U.S.C. SECTION 552.
# //  FURTHERMORE, THIS INFORMATION IS PROHIBITED FROM DISCLOSURE UNDER
# //  THE TRADE SECRETS ACT, 18 U.S.C. SECTION 1905.
# //
# Loading sv_std.std
# Loading work.orgate_tb(fast)
# Loading std.standard
# Loading std.textio(body)
# Loading ieee.std_logic_1164(body)
# Loading work.orgate(behavioral)#1
# do sim.do
# .
# waveform.wlf
# End time: 16:02:10 on Oct 03,2015, Elapsed time: 0:00:00
# Errors: 0, Warnings: 0

\end{verbatim}

\section{Waveform Screenshots}

\subsection{Questasim}

\begin{figure}[hb]
	\caption{Waveform of OR-Gate in Questasim}
	\centering
			\includegraphics[width=1.0\textwidth]{figs/questasim.png} %to change the size of figure modify the value of 0.4
				\label{fig:questasim}
\end{figure}

\subsection{Cadence Incisive}

\begin{figure}[hb]
	\caption{Waveform of OR-Gate in Cadence Incisive}
	\centering
			\includegraphics[width=1.0\textwidth]{figs/incisive.png} %to change the size of figure modify the value of 0.4
				\label{fig:incisive}
\end{figure}



\end{document}
